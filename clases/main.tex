\documentclass{apuntes}

\usepackage{lipsum}

% Detalles
\title{Nombre del Ramo}
\author{Pablo Navarro Y.}
\class{MAT0000}
\profesor{Nombre1 Apellido1}
\emailprofesor{mail@dominio.cl}
\ayudante{Nombre2 Apellido2}
\emailayudante{mail@dominio.cl}
\date{0er semestre, 2023}

% Opciones
\imprimir{}

\begin{document}
\renewcommand{\onlyinsubfile}[1]{}
\renewcommand{\notinsubfile}[1]{#1}

\maketitle

\section{Nombre sección}
\begin{definicion}[][reff]
  Definición de un punto
  \lipsum[1]
\end{definicion}
\Cref{def:reff}

\begin{prop}[][reff]
  Proposición de un punto
  \lipsum[1]
\end{prop}
\Cref{prop:reff}

\begin{teorema}[][reff]
  Teorema de un punto
  \lipsum[1]
\end{teorema}
\Cref{teo:reff}

\begin{corolario}[][reff]
  Corolario de un punto
  \lipsum[1]
\end{corolario}
\Cref{coro:reff}

\begin{lema}[][reff]
  Lema de un punto
  \lipsum[1]
\end{lema}
\Cref{lema:reff}

\begin{aff}[]
  Afirmación de un punto
  \lipsum[1]
\end{aff}

\begin{dem}[]
  Demostración de un punto
  \lipsum[1]
\end{dem}

\begin{sol}[]
  Solución de un punto
  \lipsum[1]
\end{sol}

\begin{ejemplo}[][reff]
  Ejemplo de un punto
  \lipsum[1]
\end{ejemplo}
\Cref{ej:reff}

\begin{ejercicio}[][reff]
  Ejercicio de un punto
  \lipsum[1]
\end{ejercicio}
\Cref{ejer:reff}

\begin{pregunta}[][reff]
  Pregunta de un punto
  \lipsum[1]
\end{pregunta}
\Cref{preg:reff}

\begin{obs}[][reff]
  Observación de un punto
  \lipsum[1]
\end{obs}
\Cref{obs:reff}

\begin{notacion}[]
  Notación de un punto
  \lipsum[1]
\end{notacion}

\begin{recuerdo}[]
  Recuerdo de un punto
  \lipsum[1]
\end{recuerdo}

\begin{nota}[]
  Nota de un punto
  \lipsum[1]
\end{nota}
\end{document}
