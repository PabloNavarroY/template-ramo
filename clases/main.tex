\documentclass[toc,todo]{apuntes}
\usepackage{lipsum}

\usepackage{math-macros}

% Detalles
\title{Topología y Geometría de Variedades}
\author{Pablo Navarro Y.}
\class{MAT0000}
\profesor{Nombre1 Apellido1}
\emailprofesor{mail1@dominio.cl}
\ayudante{Nombre2 Apellido2}
\emailayudante{mail2@dominio.cl}
\date{0er semestre, 2023}

\begin{document}
\renewcommand{\onlyinsubfile}[1]{}
\renewcommand{\notinsubfile}[1]{#1}
\maketitle

\section{Flag environments}
\begin{align}
  a b c d e f g h i j k l m n o p q r s t u v w x y z \\
  A B C D E F G H I J K L M N O P Q R S T U V W X Y Z \\
  \mathbb{a b c d e f g h i j k l m n o p q r s t u v w x y z} \\
  \mathbb{A B C D E F G H I J K L M N O P Q R S T U V W X Y Z} \\
  \mathcal{A B C D E F G H I J K L M N O P Q R S T U V W X Y Z} \\
  \mathfrak{a b c d e f g h i j k l m n o p q r s t u v w x y z} \\
  \mathfrak{A B C D E F G H I J K L M N O P Q R S T U V W X Y Z} \\
  \alpha \beta \gamma \delta \epsilon \varepsilon \zeta \eta \theta \vartheta \iota \kappa \lambda \mu \nu \xi \pi \rho \varrho \sigma \tau \upsilon \phi \varphi \chi \psi \omega \\
  \Gamma \Delta \Theta \Lambda \Xi \Pi \Sigma \Upsilon \Phi \Psi \Omega \\
  \nabla \partial \ell \int_{a}^{b} \oint_{\gamma} \oiint  \iint_{\Omega} \biguplus_{i=0}^{\infty} \sum_{i=0}^{\infty}\\
  \forall \exists \nexists \in \ni \notin \subset \supset \mid \wedge \vee \therefore \because \mapsto \to \gets \emptyset \implies \Rightarrow \Leftarrow \neg
  \sqrt{abc}
\end{align}
\begin{equation}
  \left\lbrace
	\begin{aligned}
	  g(y) &= \left( \int_{a}^{b}f(x)\,dx \right)\\
	  h(y) &= \frac{d\mu}{dx}(y)
	\end{aligned}
  \right.
\end{equation}
Texto sin sentido alrededor de expresiones matemáticas grandes como \(g(y) = \int_{a}^{b}f(x)\,dx\) y otras expresiones como \(h(y) = \frac{df}{dx}(y)\).

\begin{definicion}[][reff]
	\begin{equation}
		\begin{sistema}
			g(y) &= \int_{a}^{b}f(x)\,dx\\
			h(y) &= \frac{d\mu}{dx}(y)
		\end{sistema}
	\end{equation}
	Texto sin sentido alrededor de expresiones matemáticas como \(g(y) = \int_{a}^{b}f(x)\,dx\) y otras expresiones como \(h(y) = \frac{df}{dx}(y)\).
	\lipsum[1]
\end{definicion}
\Cref{def:reff}

\begin{prop}[][reff]
	\lipsum[1]
\end{prop}
\Cref{prop:reff}

\begin{teorema}[][reff]
	\lipsum[1]
\end{teorema}
\Cref{teo:reff}

\begin{corolario}[][reff]
	\lipsum[1]
\end{corolario}
\Cref{coro:reff}

\begin{lema}[][reff]
	\lipsum[1]
\end{lema}
\Cref{lema:reff}

\section{Demostration environments}

\begin{dem}[]
	\lipsum[1]
\end{dem}

\begin{sol}[]
	\lipsum[1]
\end{sol}

\section{Plain environments}

\begin{aff}[]
	\lipsum[1]
\end{aff}

\begin{ejemplo}[][reff]
	\lipsum[1]
\end{ejemplo}
\Cref{ej:reff}

\begin{ejercicio}[][reff]
	\lipsum[1]
\end{ejercicio}
\Cref{ejer:reff}

\begin{pregunta}[][reff]
	\lipsum[1]
\end{pregunta}
\Cref{preg:reff}

\begin{obs}[][reff]
	\lipsum[1]
\end{obs}
\Cref{obs:reff}

\begin{notacion}[]
	\lipsum[1]
\end{notacion}

\begin{recuerdo}[]
	\lipsum[1]
\end{recuerdo}

\begin{nota}[]
	\lipsum[1]
\end{nota}
\end{document}
